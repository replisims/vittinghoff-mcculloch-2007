\documentclass[10,a4paperpaper,]{article}

  \title{Replication study of Vittinghof and McCulloch's (2007) simulation study.}
  \author{Rick Nijman Jolien Ketelaar \textsuperscript{1}, \and Lieke Hesen \textsuperscript{2}}
  \date{%
		\textsuperscript{1} \textasciitilde{}\\%
		\textsuperscript{2} \textasciitilde{}~\\[2ex]
		\today
   }
  


\newcommand{\iblue}{008080}
\newcommand{\igray}{d4dbde}

% Author: Karol KozioL
% License: GPL-3
% Modified by: Sarah Wagner & Anna Lohmann

% % % packages -----------------------------------------------------------------------------------
\usepackage{amsmath}
\usepackage{array}
\usepackage{booktabs}
\usepackage{calc}
\usepackage{eso-pic}
\usepackage{fancyhdr}
\usepackage{fontspec}
\usepackage[left = 2.5cm, right = 2.5cm, top = 1.2cm, bottom = 1.2cm, includeheadfoot]{geometry}
\usepackage{graphicx}
\usepackage[utf8]{inputenc}
\usepackage{lastpage}
\usepackage{multirow}
\usepackage{tabularx} 
\usepackage{tikz}
\usepackage{titlesec}
\usepackage{xcolor, colortbl}
\usepackage{url} 
\usepackage[hidelinks]{hyperref} 
\usepackage{pmboxdraw}
\usepackage{placeins}
\usepackage{enumitem}
\usepackage{longtable}
\usepackage{lscape}
%\usepackage{verbatim}
%\makeatletter
%\def\verbatim@font{\scriptsize\ttfamily}
%\makeatother
%\RequirePackage[normalem]{ulem} %DIF PREAMBLE
%\RequirePackage{color}
%\providecommand{\tightlist}{%
%	\setlength{\itemsep}{0pt}\setlength{\parskip{0pt}}
\providecommand{\tightlist}{%
  \setlength{\itemsep}{0pt}\setlength{\parskip}{0pt}}
% % % settings -----------------------------------------------------------------------------------

% % custom colors
\definecolor{iblue}{HTML}{\iblue}
\definecolor{igray}{HTML}{\igray}

% definition of pagename
\newcommand\pagename{Page}

% % fonts 
\defaultfontfeatures{Mapping = tex-text}
\setmainfont[BoldFont = Lato-Bold.ttf, ItalicFont = Lato-Italic.ttf, BoldItalicFont = Lato-BoldItalic.ttf]{Lato-Regular.ttf}
\newfontfamily\headingfont[ItalicFont = Lato-BlackItalic.ttf]{Lato-Black.ttf}
%\setmonofont{Ubuntu Mono}
\setmonofont[Scale=0.90,
BoldFont=UbuntuMono-Bold.ttf,
%ItalicFont=UbuntuMono-Italic.ttf,
BoldItalicFont=UbuntuMono-BoldItalic.ttf
]{UbuntuMono-Regular.ttf}

\makeatletter
\def\verbatim@font{\linespread{1}\normalfont\ttfamily}
\makeatother

% % sections
\titleformat{\section}{\color{iblue}\headingfont\Large\bfseries}{\thesection}{1em}{}[\titlerule]
\titleformat{\subsection}{\color{iblue}\headingfont\large\bfseries}{\thesubsection}{1em}{}
\titleformat{\subsubsection}{\color{iblue}\headingfont\bfseries}{\thesubsubsection}{1em}{}

% % misc
\setlength{\parindent}{0em} 
\linespread{1.5}
\raggedright
\newcolumntype{C}{>{\centering\arraybackslash}X}


\makeatletter

% pagestyle titlepage
\fancypagestyle{customtitle}{
	\lhead{}
	\chead{}
	\rhead{}
	\makeatother
	\lfoot{}
	\cfoot{}
	\rfoot{}
}




% % % header and footer ---------------------------------------------------------------------------
\pagestyle{fancy}
\lhead{}
\chead{}
\rhead{}
\makeatother
\newlength{\myheight}
\lfoot{}
\cfoot{}
\rfoot{\pagename~\thepage \hspace{1pt} / \pageref{LastPage}}
\renewcommand\headrulewidth{0pt}
\renewcommand\footrulewidth{0pt}




\begin{document}


\renewcommand{\contentsname}{Table of Contents}

\renewcommand{\pagename}{Page}


\urlstyle{same}

\maketitle

\subsection*{Abstract}

\texttt{\textless{}a\ summary\ of\ the\ replication\ effort\textgreater{}}
\vskip 2em

\noindent\makebox[\textwidth]{\large Correspondence concerning this replication report should be addressed to:}

\par

\noindent\makebox[\textwidth]{\large ...}

\par

\clearpage

\section{Introduction}

This replication report documents the replication attempt of the
simulation study of Vittinghoff and McCulloch (2007). Following the
definition of Rougier et al. (2017) we understand the replication of a
published study as writing and running new code based on the description
provided in the original publication with the aim of obtaining the same
results.

\section{Method}

\subsection{Information basis}

The original article is solely used as an information basis.

\subsection{Data Generating Mechanism}

Information provided in the above mentioned sources indicated that the
following simulation factors were systematically varied in generating
the artificial data.

\begin{longtable}[]{@{}lll@{}}
\toprule
\begin{minipage}[b]{0.37\columnwidth}\raggedright
Simulation factor\strut
\end{minipage} & \begin{minipage}[b]{0.12\columnwidth}\raggedright
No.~levels\strut
\end{minipage} & \begin{minipage}[b]{0.43\columnwidth}\raggedright
Levels\strut
\end{minipage}\tabularnewline
\midrule
\endhead
\begin{minipage}[t]{0.37\columnwidth}\raggedright
\emph{General}\strut
\end{minipage} & \begin{minipage}[t]{0.12\columnwidth}\raggedright
\strut
\end{minipage} & \begin{minipage}[t]{0.43\columnwidth}\raggedright
\strut
\end{minipage}\tabularnewline
\begin{minipage}[t]{0.37\columnwidth}\raggedright
Events per variable\strut
\end{minipage} & \begin{minipage}[t]{0.12\columnwidth}\raggedright
15\strut
\end{minipage} & \begin{minipage}[t]{0.43\columnwidth}\raggedright
2,3,4,5,6,7,8,9,10,11,12,13,14,15,16\strut
\end{minipage}\tabularnewline
\begin{minipage}[t]{0.37\columnwidth}\raggedright
Number of predictors\strut
\end{minipage} & \begin{minipage}[t]{0.12\columnwidth}\raggedright
4\strut
\end{minipage} & \begin{minipage}[t]{0.43\columnwidth}\raggedright
2,4,8,16\strut
\end{minipage}\tabularnewline
\begin{minipage}[t]{0.37\columnwidth}\raggedright
Sample sizes\strut
\end{minipage} & \begin{minipage}[t]{0.12\columnwidth}\raggedright
4\strut
\end{minipage} & \begin{minipage}[t]{0.43\columnwidth}\raggedright
128, 256, 512, 1024\strut
\end{minipage}\tabularnewline
\begin{minipage}[t]{0.37\columnwidth}\raggedright
B1 (coefficient primary predictor)\strut
\end{minipage} & \begin{minipage}[t]{0.12\columnwidth}\raggedright
4\strut
\end{minipage} & \begin{minipage}[t]{0.43\columnwidth}\raggedright
0,log(1.5),log(2),log(4)\strut
\end{minipage}\tabularnewline
\begin{minipage}[t]{0.37\columnwidth}\raggedright
Pairwise correlation (with binary predictor of other predictors)\strut
\end{minipage} & \begin{minipage}[t]{0.12\columnwidth}\raggedright
1 (fixed)\strut
\end{minipage} & \begin{minipage}[t]{0.43\columnwidth}\raggedright
0.25\strut
\end{minipage}\tabularnewline
\begin{minipage}[t]{0.37\columnwidth}\raggedright
\emph{Specific for binary primary predictor}\strut
\end{minipage} & \begin{minipage}[t]{0.12\columnwidth}\raggedright
\strut
\end{minipage} & \begin{minipage}[t]{0.43\columnwidth}\raggedright
\strut
\end{minipage}\tabularnewline
\begin{minipage}[t]{0.37\columnwidth}\raggedright
Expected prevalence of primary predictor\strut
\end{minipage} & \begin{minipage}[t]{0.12\columnwidth}\raggedright
3\strut
\end{minipage} & \begin{minipage}[t]{0.43\columnwidth}\raggedright
0.1, 0.25, 0.5\strut
\end{minipage}\tabularnewline
\begin{minipage}[t]{0.37\columnwidth}\raggedright
Multiple correlation of primary predictor with covariates\strut
\end{minipage} & \begin{minipage}[t]{0.12\columnwidth}\raggedright
4\strut
\end{minipage} & \begin{minipage}[t]{0.43\columnwidth}\raggedright
0,0.25,0.5,0.75\strut
\end{minipage}\tabularnewline
\begin{minipage}[t]{0.37\columnwidth}\raggedright
\emph{Specific for continuous primary predictor}\strut
\end{minipage} & \begin{minipage}[t]{0.12\columnwidth}\raggedright
\strut
\end{minipage} & \begin{minipage}[t]{0.43\columnwidth}\raggedright
\strut
\end{minipage}\tabularnewline
\begin{minipage}[t]{0.37\columnwidth}\raggedright
Variance primary predictor\strut
\end{minipage} & \begin{minipage}[t]{0.12\columnwidth}\raggedright
1 (fixed)\strut
\end{minipage} & \begin{minipage}[t]{0.43\columnwidth}\raggedright
0.16\strut
\end{minipage}\tabularnewline
\begin{minipage}[t]{0.37\columnwidth}\raggedright
Multiple correlation of primary predictor with covariates\strut
\end{minipage} & \begin{minipage}[t]{0.12\columnwidth}\raggedright
5\strut
\end{minipage} & \begin{minipage}[t]{0.43\columnwidth}\raggedright
0,0.1,0.25, 0.5, 0.9\strut
\end{minipage}\tabularnewline
\bottomrule
\end{longtable}

\subsubsection{Events per variable}

``We considered values for the EPV from two to 16{[}\ldots{]}'' (p.~710)
althus the authors.

\subsubsection{Number of predictors}

Four different levels are used for the number of predictors, namely
``{[}\ldots{]} models with a total of two, four, eight and 16 predictor
variables {[}\ldots{]}'' (p.~710).

\subsubsection{Sample sizes}

The authors describe the sample sizes as follows: ``{[}\ldots{]} sample
sizes of 128, 256, 512 and 1,024 {[}\ldots{]}'' (p.710).

\subsubsection{B1 (coefficient primary predictor)}

The regression coefficients of B1, from the primary predictor on the
outcome variable range from 0 to 0.75, as described by: ``{[}\ldots{]}
values of B1, the regression coefficient for the primary predictor, of
0, 0.25, 0.5, or 0.75 {[}\ldots{]}'' (p.710).

\subsubsection{Pairwise correlation (with binary predictor of other predictors)}

The authors state that: ``With a binary primary predictor, the other
predictors were multivariate normal with pairwise correlation of 0.25.''
(p.~710). Later on the authors state that: ``With the continuous primary
predictor, all predictors were multivariate normal and equally
intercorrelated.'' (p.~711). Therefore, we assumed that the pairwise
correlation of 0.25 is also used for the continuous predictor and is not
varied over scenarios.

\subsubsection{Expected prevalence}

``The binary primary predictor was generated with expected prevalence of
0.1, 0.25, 0.5, 0.75.'' (p.~710).

\subsubsection{Multiple correlation of primary predictor with covariates}

The authors state that: ``The binary predictor was generated with
{[}\ldots{]}, and multiple correlation with the covariates of 0, 0.25,
0.5 and 0.75'' (p.~711). Later on the authors state that: "{[}\ldots{]}
for comparability with the binary predictors, and the multiple
correlation between the primary predictor and adjustment variables was
set to 0, 0.1, 0.25, 0.5 or 0.9 (p.711- p.712). The authors are first
talking about covariates and later about adjustment variables. We
interpreted this as the same concept: the correlation between primary
predictor and covariance matrix.

\subsubsection{Variance primary predictor and covariates}

Vittinghof and McCulloch (2007) describe ``The variance of the
continuous primary predictor was set to 0.16'' (p.~710). This was
constant over all conditions with the continuous primary predictor.
Neither the variance for the binary primary predictor, nor the variance
of the covariates was described.

\subsubsection{Omit extreme cases}

Vittinghof and McCulloch (2007) state that: ``The factorial omitted
extreme cases with outcome prevalence of greater than 50 percent''
(p.710).\\
We assumed that the authors did this in the following way. In each
factor the number of predictors and the number of events per variable is
known. The number of predictors multiplied by the events per variable is
the number of events per factor. Also, the sample size is determined
beforehand. The number of events divided by the sample size is the
outcome prevalence. Each factor has its own outcome prevalence, when the
outcome prevalence was higher than 0.5, we filtered out the cases.

This method leads to 10176 for the binary predictor and 4240 for the
continuous, where the authors examined 9328 and 3392 scenarios (for
details on the calculation the code is included on github).

\subsection{Data generating mechanism}

\subsubsection{The aggregate effect}

Vittinghof and McCulloch (2007) state that: ``The aggregate effect of
the covariates is held constant across models with two, four, eight and
16 predictors'' (p.~711). However, the aggregate effect is not
specified.

\subsubsection{Binary outcome data}

\subsubsection{Retrospective sampling}

Vittinghof and McCulloch (2007) state that: ``In logistic models, we
kept the first ``cases'' and ``controls'' generated up to the required
numbers of each, taking advantage of the fact that under the logistic
model only the intercept is affected by such retrospective sampling''
(p.~710-711). It is unclear how this procedure is conducted. We assumed
that the authors oversampled cases and controls and kept the required
first cases and controls. Unfortunately, it was not defined how many
cases and controls were oversampled.

\subsubsection{Logistic model with binary predictor}

Vittinghof and McCulloch (2007) are not giving information about how
they generated the data for the logistic model with the binary
predictor. It is unclear how the binary primary predictor with the
predefined prevalence is generated while keeping the predefined
correlation structure intact.

\subsubsection{Time-to-event data}

Vittinghof \& McCulloch (2007) provide rather limited data on how the
data for the cox model was generated. It is unclear which distribution
underlies the survival analysis. One can guess by looking back at
earlier mentioned articles, such as Peduzzi et al.~(1995). However
Vittinghof \& McCulloch do not state they follow a similar procedure.
Not knowing which distribution (e.g.~Weibull) makes the simulation
impossible to replicate.

\subsubsection{Additional simulations}

The original authors describe that they also ``examined models with all
binary predictors'' (p.717), again with a logistic and Cox model. They
did not provide any parameters they used in these additional
simulations.

\subsection{Compared Methods}

NA. We did not compare methods since we were unable to generate the
data.

\subsection{Performance measures}

NA. We did not evaluate performance measures since we were unable to
generate the data.

\subsection{Technical implementation}

NA. We did not implement the simulation study since we were unable to
generate the data.

\section{Results}

NA. No results were obtained since we were unable to generate the data.
We did find it striking that in Table 1 of the original article, the
percentage for the problem of a relative bias from above 15 percent was
higher than the percentage of any of the three problems for more than
half of the scenarios. This probably has to do with the exclusion of
some scenarios for which the decision criteria were not described.
Furthermore, Table 2 of the original article shows percentages of higher
than 100\%, e.g.~260.1 for 2-4 events per variable for the problem of
maximum relative bias. Here, an explanation is missing.

\subsection{Replicability}

The paper of Vittinghof and McCulloch (2007) is not replicable. We
cannot recreate the factorial as described in the paper, because the
data generating parameters are insufficiently described. We attempted to
guess what the intended set-up of the simulation was. Nevertheless, we
attempted to recreate the intended set-up by making a best guess of the
factorial. This resulted in 10176 and 4240 scenarios for the binary and
continuous primary predictors, whereas the authors of the original paper
mention 9328 and 3392 scenarios. Even when making this best guess, many
subsequent factors were still unknown, such as the underlying model of
the Cox regression. In conclusion, the provided information was so
scarce that the replicator degrees of freedom were too large to perform
any of the subsequent steps of the analysis.

\subsection{Replicator degrees of freedom}

We were not able to replicate this study because the replicator degrees
of freedom tended to go to infinity. The table below specifies the
issue, our decision and justification if applicable.

\begin{longtable}[]{@{}lll@{}}
\toprule
\begin{minipage}[b]{0.30\columnwidth}\raggedright
\emph{Issue}\strut
\end{minipage} & \begin{minipage}[b]{0.30\columnwidth}\raggedright
\emph{Replicator decision}\strut
\end{minipage} & \begin{minipage}[b]{0.30\columnwidth}\raggedright
\emph{Justification}\strut
\end{minipage}\tabularnewline
\midrule
\endhead
\begin{minipage}[t]{0.30\columnwidth}\raggedright
Variance binary primary predictor and covariates unknown\strut
\end{minipage} & \begin{minipage}[t]{0.30\columnwidth}\raggedright
We decided to use a variance of 1.\strut
\end{minipage} & \begin{minipage}[t]{0.30\columnwidth}\raggedright
A variance of 1 seemed the most convenient choice, because this means
the covariance matrix is equivalent to the correlation matrix. This
choice, however, resulted in some covariance matrices that were
non-positive definite\strut
\end{minipage}\tabularnewline
\begin{minipage}[t]{0.30\columnwidth}\raggedright
Generation of correlation binary primary predictor with covariates while
keeping the multivariate normal covariance structure\strut
\end{minipage} & \begin{minipage}[t]{0.30\columnwidth}\raggedright
We used a logistic regression model, where we tried to find beta values
such that the correlation would resemble the described values.\strut
\end{minipage} & \begin{minipage}[t]{0.30\columnwidth}\raggedright
To our knowledge, this was the only way to keep the overall covariance
structure. We didn't know a way to simulate both binary and continuous
variables given a certain covariance matrix. We were not able to find a
beta value to get to a correlation of 0.75 between the primary predictor
and covariates.\strut
\end{minipage}\tabularnewline
\begin{minipage}[t]{0.30\columnwidth}\raggedright
The aggregate effect of the covariates was unknown\strut
\end{minipage} & \begin{minipage}[t]{0.30\columnwidth}\raggedright
This unknown was not resolved\strut
\end{minipage} & \begin{minipage}[t]{0.30\columnwidth}\raggedright
-\strut
\end{minipage}\tabularnewline
\begin{minipage}[t]{0.30\columnwidth}\raggedright
The underlying distribution of the survival analysis is unknown\strut
\end{minipage} & \begin{minipage}[t]{0.30\columnwidth}\raggedright
The decision was made to stop the replication study\strut
\end{minipage} & \begin{minipage}[t]{0.30\columnwidth}\raggedright
We could have looked back at older papers, however the authors do not
specify they followed the same procedure. The amount of degrees of
freedom lead to the decision to stop the replication\strut
\end{minipage}\tabularnewline
\begin{minipage}[t]{0.30\columnwidth}\raggedright
The pairwise correlation of the continuous primary predictor was not
explicitly specified\strut
\end{minipage} & \begin{minipage}[t]{0.30\columnwidth}\raggedright
We assume the same pairwise correlation as was specified for the binary
predictor\strut
\end{minipage} & \begin{minipage}[t]{0.30\columnwidth}\raggedright
The authors state that: ``With the continuous primary predictor, all
predictors were multivariate normal and equally intercorrelated.''
(p.~711). Therefore, we assumed that the pairwise correlation of 0.25 is
also used for the continuous predictor and is not varied over
scenarios.\strut
\end{minipage}\tabularnewline
\begin{minipage}[t]{0.30\columnwidth}\raggedright
The authors use both the term covariates and the term adjustment
variables\strut
\end{minipage} & \begin{minipage}[t]{0.30\columnwidth}\raggedright
We interpreted this as the same concept: the correlation between primary
predictor and covariance matrix.\strut
\end{minipage} & \begin{minipage}[t]{0.30\columnwidth}\raggedright
The authors do not specify which is meant by adjustment variables
otherwise\strut
\end{minipage}\tabularnewline
\begin{minipage}[t]{0.30\columnwidth}\raggedright
The way the factorial omitted extreme cases is unclear\strut
\end{minipage} & \begin{minipage}[t]{0.30\columnwidth}\raggedright
We assumed that the authors did this in the following way. In each
factor the number of predictors and the number of events per variable is
known. The number of predictors multiplied by the events per variable is
the number of events per factor. Also, the sample size is determined
beforehand. The number of events divided by the sample size is the
outcome prevalence. Each factor has its own outcome prevalence, when the
outcome prevalence was higher than 0.5, we filtered out the cases.\strut
\end{minipage} & \begin{minipage}[t]{0.30\columnwidth}\raggedright
This made the most sense\strut
\end{minipage}\tabularnewline
\bottomrule
\end{longtable}

\subsection{Equivalence of results}

NA. Results were not obtained since we were unable to generate the data.

\section{Acknowledgments}

We want to acknowledge Kim Luijken and David Koops for their help during
the difficult steps of this replication study. They provided us with
insights and tips, which inspired us to try and replicate the study.
Unfortunately their enthusiasm did not help us do the impossible:
replicate the study.

\section{References}

Vittinghoff, E., \& McCulloch, C. E. (2007). Relaxing the rule of ten
events per variable in logistic and Cox regression. American journal of
epidemiology, 165(6), 710-718.

Peduzzi P, Concato J, Feinstein AR, et al.~Importance of events per
independent variable in proportional hazards regression analysis. II.
Accuracy and precision of regression estimates. J Clin Epidemiol
1995;48:1503--10

\hypertarget{refs}{}
\leavevmode\hypertarget{ref-rougier_sustainable_2017-1}{}%
Rougier, Nicolas P., Konrad Hinsen, Frédéric Alexandre, Thomas Arildsen,
Lorena A. Barba, Fabien C. Y. Benureau, C. Titus Brown, et al. 2017.
``Sustainable Computational Science: The ReScience Initiative.''
\emph{PeerJ Computer Science} 3 (December): e142.
\url{https://doi.org/10.7717/peerj-cs.142}.


\end{document}
